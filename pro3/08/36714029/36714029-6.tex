\documentclass[a4j,11pt]{jarticle}
\usepackage{fancyheadings,listings}
%\usepackage{here}
\lstset{
  basicstyle={\ttfamily},
  identifierstyle={\small},
  commentstyle={\smallitshape},
  keywordstyle={\small\bfseries},
  ndkeywordstyle={\small},
  stringstyle={\small\ttfamily},
  frame={tb},
  breaklines=true,
  columns=[l]{fullflexible},
  numbers=left,
  xrightmargin=0zw,
  xleftmargin=3zw,
  numberstyle={\scriptsize},
  stepnumber=1,
  numbersep=1zw,
  lineskip=-0.5ex
}
\setlength{\headheight}{15.5pt}
\usepackage[dvipdfmx]{graphicx, color}
\renewcommand{\thepage}{\small -- \arabic{page} --}
%\headrulewidth=0pt
\rhead{}
\lhead{}
\textheight 234mm
%
\usepackage{amsmath}	% required for `\cases' (yatex added)
\usepackage{bm}
\usepackage{ascmac}	% required for `\screen' (yatex added)

\newcommand{\thisyear}{2019}

\pagestyle{fancy}
\lhead{2025$BG/EY(B $B%W%m%0%i%_%s%0(BIII}

\title{2025$BG/EY(B $B%W%m%0%i%_%s%0(BIII $BBh(B1$B2s(B $B%l%]!<%H(B}
\date{2025 $BG/(B10$B7n(B2$BF|(B}

\author{学籍番号36714029 \\ 遠藤裕人}
%
%
\begin{document}
%
\maketitle
\clearpage
%
%
\section{$B$O$8$a$K(B}
%
%
$B1i=,2]Bj(B1$B$N<B9T7k2L$K$D$$$FJs9p$7$^$9!#(B
%
%
\section{$B1i=,2]Bj(B}
%
%
\subsection{2.1 課題8}

題名:何を作ったか
初めてC言語とM5Stick CPlus1.1を使ってマイコンを振るとLEDが点灯する機能を作成した。

どのような技術を使ったか(条件)
C言語、M5Stick CPlus1.1、M5Stackライブラリ、Arduino

実行例

プログラムの解説

コンパイル方法(ライブラリ等)

プログラムの動作(自作関数の入出力や中身の説明)について

感想
苦戦したところとして、
まずほかにM5StickCPlus2やM5Stackなどの紛らわしい名前のマイコンがあるせいで対応するボードやライブラリの捜索が困難だった。\\
また依存関係のインストールがnpmインストールの5倍くらい時間がかかった。\\
USBポートの指定が間違っていたためマイコンをPCが認識しなかった。\\

\begin{lstlisting}[caption=hoge. label=fuge]
    #include <stdio.h>
    #define NAME_LEN 64
    struct student{
        char name[NAME_LEN];
        int height;
        double weight;
    };
    int main(void){
        struct student takao = {"Takao", 173, 70.5};
    
        printf("\n--- 各メンバのアドレス ---\n");
        printf("takao.nameのアドレス   = %p\n", (void*)&takao.name);
        printf("takao.heightのアドレス = %p\n", (void*)&takao.height);
        printf("takao.weightのアドレス = %p\n", (void*)&takao.weight);
    
        return 0;
    }
    
  
\end{lstlisting}
出力:\\
\includegraphics[height=30mm, width=100mm]{task6-1-output.png}


学籍番号:36714029 氏名: 遠藤裕人
\end{document}