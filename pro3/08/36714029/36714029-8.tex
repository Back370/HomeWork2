\documentclass[a4j,11pt]{jarticle}
\usepackage{fancyheadings,listings}
%\usepackage{here}
\lstset{
  basicstyle={\ttfamily},
  identifierstyle={\small},
  commentstyle={\smallitshape},
  keywordstyle={\small\bfseries},
  ndkeywordstyle={\small},
  stringstyle={\small\ttfamily},
  frame={tb},
  breaklines=true,
  columns=[l]{fullflexible},
  numbers=left,
  xrightmargin=0zw,
  xleftmargin=3zw,
  numberstyle={\scriptsize},
  stepnumber=1,
  numbersep=1zw,
  lineskip=-0.5ex
}
\setlength{\headheight}{15.5pt}
\usepackage[dvipdfmx]{graphicx, color}
\renewcommand{\thepage}{\small -- \arabic{page} --}
%\headrulewidth=0pt
\rhead{}
\lhead{}
\textheight 234mm
%
\usepackage{amsmath}	% required for `\cases' (yatex added)
\usepackage{bm}
\usepackage{ascmac}	% required for `\screen' (yatex added)

\newcommand{\thisyear}{2019}

\pagestyle{fancy}
\lhead{2025$BG/EY(B $B%W%m%0%i%_%s%0(BIII}

\title{2025$BG/EY(B $B%W%m%0%i%_%s%0(BIII $BBh(B1$B2s(B $B%l%]!<%H(B}
\date{2025 $BG/(B10$B7n(B2$BF|(B}

\author{学籍番号36714029 \\ 遠藤裕人}
%
%
\begin{document}
%
\maketitle
\clearpage
%
%
\section{$B$O$8$a$K(B}
%
%
$B1i=,2]Bj(B1$B$N<B9T7k2L$K$D$$$FJs9p$7$^$9!#(B
%
%
\section{$B1i=,2]Bj(B}
%
%
\subsection{2.1 課題8}

\subsubsection{題名:何を作ったか}
呪われたTodoリスト管理システムを作成した。このプログラムは通常のタスク管理機能に加えて、
ランダムで発生する恐怖演出(文字化け、赤色表示、不気味なメッセージ)を含む、
ホラー要素を持ったTodoリストアプリケーションである。

\subsubsection{どのような技術を使ったか}

\begin{itemize}
\item 制御構文の利用: if文、while文、switch文、for文を使用
\item 関数(自作)の利用: 20個以上の関数を定義・利用
\item ポインタの利用: 構造体ポインタを使用してタスク情報を操作
\item 構造体の利用: Task、Date、TodoListの3つの構造体を定義
\item ファイル操作: バイナリファイルによるデータの永続化
\item 250行以上: 530行のコードで実装
\end{itemize}

\subsubsection{実行例}

\subsubsection{プログラムの解説}

\begin{enumerate}
\item タスク登録機能:タスクの内容、優先度(Low/Medium/High)、期限を入力して登録
\item タスク編集機能:登録済みのタスクの内容を編集、または削除
\item 月次タスク表示:今月のタスクを表形式で一覧表示
\item 今日のタスク表示:本日期限のタスクを目次形式で表示
\item ファイル永続化:タスクデータをtask8.datファイルに保存・読み込み
\end{enumerate}

恐怖演出として以下の機能を実装:
\begin{itemize}
\item タスク登録時に10\%の確率でタスク内容が文字化け
\item 月次タスク表示時に12.5\%の確率で全ての線が赤色に変化
\item 月次タスク表示時に16.7\%の確率で不気味なタスクが追加表示
\item メニュー操作後に20\%の確率で恐怖メッセージを表示
\end{itemize}

\subsubsection{コンパイル方法}
以下のコマンドでコンパイル可能:
\begin{verbatim}
gcc task8.c -o task8
\end{verbatim}

使用ライブラリ:
\begin{itemize}
\item stdio.h: 標準入出力
\item stdlib.h: 標準ライブラリ(rand関数等)
\item string.h: 文字列操作(strlen、strcspn等)
\item time.h: 日時取得(time、localtime等)
\end{itemize}

\subsubsection{プログラムの動作(自作関数の説明)}

主要な自作関数の説明:

\begin{description}
\item[initialize\_list(TodoList* list)] ~\\
入力:TodoListへのポインタ \\
出力:なし \\
動作:Todoリストを初期化し、全タスクを非アクティブ状態にする

\item[add\_task(TodoList* list)] ~\\
入力:TodoListへのポインタ \\
出力:なし \\
動作:ユーザーからタスク情報(内容、優先度、期限)を入力し、リストに追加する。
10\%の確率でcorrupt\_text関数を呼び出し、タスク内容を文字化けさせる

\item[edit\_task(TodoList* list)] ~\\
入力:TodoListへのポインタ \\
出力:なし \\
動作:タスクIDを指定して、タスクの内容・優先度・期限を編集、または削除する

\item[display\_monthly\_tasks(TodoList* list)] ~\\
入力:TodoListへのポインタ \\
出力:なし \\
動作:現在の月と同じ月のタスクを表示する。12.5\%の確率で赤色表示になり、
16.7\%の確率で不気味なタスクが追加される

\item[save\_to\_file(TodoList* list)] ~\\
入力:TodoListへのポインタ \\
出力:なし \\
動作:タスク情報をバイナリファイル(task8.dat)に保存する

\item[load\_from\_file(TodoList* list)] ~\\
入力:TodoListへのポインタ \\
出力:なし \\
動作:task8.datからタスク情報を読み込み、リストに復元する

\item[corrupt\_text(char* text)] ~\\
入力:文字列へのポインタ \\
出力:なし \\
動作:文字列の各文字を33\%の確率でランダムな文字に置換し、文字化けを演出する

\item[is\_valid\_date(Date* date)] ~\\
入力:Dateへのポインタ \\
出力:有効な日付なら1、無効なら0 \\
動作:年・月・日の妥当性をチェックし、うるう年も考慮する
\end{description}

\subsubsection{感想}
今回のプログラム作成で苦労した点と学んだ点:

\begin{itemize}
\item 構造体とポインタを組み合わせた複雑なデータ構造の設計が難しかった。
特にTodoList構造体内にTask配列を持たせ、それをポインタで操作する部分で理解が深まった。

\item ファイル操作では、バイナリモードでの読み書きを使用することで、
構造体をそのまま保存・読み込みできることを学んだ。

\item 乱数を使った確率的な動作の実装が面白かった。
rand関数とsrand関数の使い方、特にtimeをシード値にすることで
毎回異なる乱数を生成できることを理解した。

\item 日付の妥当性チェックでは、うるう年の判定ロジックを実装し、
暦の仕組みについても理解を深めることができた。月の日数を配列で管理する方法も学んだ。

\item ANSIエスケープコードを使った色付き出力により、
コンソールでも視覚的に面白い演出ができることを知った。
\end{itemize}


学籍番号:36714029 氏名: 遠藤裕人
\end{document}