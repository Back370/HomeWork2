\documentclass[10pt,a4j]{ujarticle}
\usepackage{graphicx}
\usepackage{url}
\usepackage{listings,jlisting}
\usepackage{ascmac}
\usepackage{amsmath,amssymb}

%ここからソースコードの表示に関する設定
\lstset{
  basicstyle={\ttfamily},
  identifierstyle={\small},
  commentstyle={\smallitshape},
  keywordstyle={\small\bfseries},
  ndkeywordstyle={\small},
  stringstyle={\small\ttfamily},
  frame={tb},
  breaklines=true,
  columns=[l]{fullflexible},
  numbers=left,
  xrightmargin=0zw,
  xleftmargin=3zw,
  numberstyle={\scriptsize},
  stepnumber=1,
  numbersep=1zw,
  lineskip=-0.5ex
}
%ここまでソースコードの表示に関する設定

\title{知能処理学 レポート}
\author{
 36714029 遠藤裕人\\
  {\small (学生番号と氏名が必要)}
}
\date{\today}


\begin{document}
\maketitle

%\paragraph{提出レポート: } rep0
\paragraph{全体的な自己評価/作業時間: } S/6時間

自己評価を S, A, B, C, D から選択する.作業時間は,成績に影響しないので正直に書くこと.

\paragraph{評価の理由: } 迷路探索アルゴリズムの実装において,各処理(ゴール検査,選択,展開,生成)を独立したメソッドに分離することで可読性と保守性を向上させた.また,幅優先探索と深さ優先探索の違いを引数の順序変更のみで実現できることを示し,アルゴリズムの本質的な理解を深めることができたため.


%%%%%%%%%%%%%%%%%%%%%%%%%%%%%%%%%%%%%%%%
\section{課題1-1a}

\begin{screen}
package ex1a;

import java.util.*;
import java.util.stream.*;

public class Maze {
  public static void main(String[] args) {
  var maze = new Maze();
  maze.solve();
  }

  Map<String, List<String>> map = Map.of(
  "A", List.of("B", "C", "D"),
  "B", List.of("E", "F"),
  "C", List.of("G", "H"),
  "D", List.of("I", "J"));

  public void solve() {
  if (search("A") != null)
  System.out.println("found");
  }

  String search(String root) {
  List<String> openList = new ArrayList<>();
  openList.add(root);

  while (!openList.isEmpty()) {
  var state = get(openList);

  if (isGoal(state))
  return state;

  var children = children(state);
  openList = concat(openList, children);
  }

  return null;
  }

  boolean isGoal(String state) {
  return "E".equals(state);
  }

  String get(List<String> list) {
  return list.remove(0);
  }

  List<String> children(String current) {
  return this.map.getOrDefault(current, Collections.emptyList());
  }

  List<String> concat(List<String> xs, List<String> ys) {
  return toMutable(Stream.concat(xs.stream(), ys.stream()).toList());
  }

  List<String> toMutable(List<String> list) {
  return new ArrayList<>(list);
  }
}

\end{screen}

\subsection{実行結果} 

\begin{figure}[!hbt]
  \centering
  \includegraphics[bb=0 0 782 352,width=0.6\linewidth]{1-1result.png}
  \caption{実行結果}
  \label{fig:result}
\end{figure}

\paragraph{自己評価/作業時間: } A/1時間23分

\section{課題1-1}

\begin{screen}
  ゴール検査,選択,展開,生成の処理を実現している関数名を答えよ.
\end{screen}

\subsection{ゴール検査:} 
\begin{lstlisting}
	boolean isGoal(String state) {
		return "E".equals(state);
	}
\end{lstlisting}

\subsection{選択:} 
\begin{lstlisting}
String get(List<String> list) {
		return list.remove(0);
	}
\end{lstlisting}

\subsection{展開:}
\begin{lstlisting}
	List<String> children(String current) {
		return this.map.getOrDefault(current, Collections.emptyList());
	}
\end{lstlisting}

\subsection{生成:}
\begin{lstlisting}
List<String> concat(List<String> xs, List<String> ys) {
		return toMutable(Stream.concat(xs.stream(), ys.stream()).toList());
	}
\end{lstlisting}

\paragraph{工夫点: } 探索アルゴリズムの実装において,各機能を独立したメソッドに分離することで保守性を高めた.具体的には,ゴール検査,選択,展開,生成の各処理を明確に分離し,アルゴリズムの理解と変更を容易にした.

\section{課題1-2}

\begin{screen}
  このプログラムが実現している探索手法を答えよ.その根拠となる部分をレポートに転記しその動作を説明せよ.
\end{screen}

横型探索を採用している。
下記のプログラムのように子ノードを後に追加している。つまり親ノードをすべて探索してから子ノードを探索しようとしている

\begin{lstlisting}
var children = children(state);
			openList = concat(openList, children);
\end{lstlisting}

\section{課題1-3}

\begin{screen}
  リスト 1 を縦型探索もしくは横型探索に変更するための方法を答えよ.
\end{screen}

\begin{lstlisting}
var children = children(state);
			openList = concat(children, openList);
\end{lstlisting}
このようにconcatの引数の順序を変更することで縦型探索に変更できる.


\subsection{考察・感想}

\paragraph{考察: }
今回実装した迷路探索プログラムでは,幅優先探索(横型探索)アルゴリズムを採用した.最初は深さ優先探索の方が実装が簡単だと予想していたが,実際にプログラムを作成すると幅優先探索の方がより直感的で理解しやすい実装となった.特に,openListに子ノードを後ろに追加する処理(concat(openList, children))により,同じ深さのノードを先に探索する動作が自然に実現できた.

プログラムの構造について分析すると,各処理が独立したメソッドとして分離されており,保守性が高い設計になっている.ゴール検査(isGoal),選択(get),展開(children),生成(concat)の各処理が明確に分離されているため,アルゴリズムの変更や拡張が容易である.

また,深さ優先探索への変更も,concat の引数順序を変更するだけで実現できることが分かった.具体的には,concat(children, openList) とすることで,新しく生成されたノードが優先的に選択され,深さ優先探索となる.この柔軟性は,オブジェクト指向設計の利点を活かした結果と考えられる.

性能面では,この実装は小規模な迷路に対しては十分な性能を示すが,大規模な迷路では訪問済みノードの管理が必要になると考えられる.現在の実装では同じノードを複数回訪問する可能性があり,無限ループのリスクも存在する.改善策として,Set<String>を用いた訪問済みノード管理の導入が有効であろう.

\paragraph{感想: }
探索アルゴリズムの実装を通じて,理論で学んだ内容を実際のコードで表現することの面白さと難しさを実感した.特に,アルゴリズムの動作を具体的なデータ構造(List,Map)で表現する過程で,抽象的な概念を具体化する重要性を学んだ.また,わずかなコードの変更(引数の順序)でアルゴリズムの性質が大きく変わることに驚きを感じた.今後は,より複雑な探索問題や最適化問題にも挑戦し,アルゴリズムの理解を深めていきたい.

%%%%%%%%%%%%%%%%%%%%%%%%%%%%%%%%%%%%%%%%
\section{課題1-1b}

\begin{screen}
package ex1b;

import java.util.*;

/**
 * 迷路探索問題のメインクラス
 * 幅優先探索を使用してゴール("E")までの経路を探索する
 */
public class MazeProblem {
	/**
	 * メインメソッド
	 * Solverを使用して迷路探索を実行する
	 */
	public static void main(String[] args) {
		var solver = new Solver();
		// スタート地点"A"から探索を開始
		solver.solve(new MazeWorld("A"));
	}
}

/**
 * 迷路での移動アクションを表すクラス
 * Actionインターフェースを実装
 */
class MazeAction implements Action {
	String next; // 移動先の位置

	/**
	 * コンストラクタ
	 * @param next 移動先の位置
	 */
	MazeAction(String next) {
		this.next = next;
	}

	/**
	 * アクションの文字列表現を返す
	 * @return "move to [移動先]"の形式の文字列
	 */
	public String toString() {
		return "move to " + this.next;
	}
}

/**
 * 迷路の世界状態を表すクラス
 * Worldインターフェースを実装
 */
class MazeWorld implements World {
	// 迷路のマップ構造(各位置から移動可能な位置のリスト)
	Map<String, List<String>> map = Map.of(
			"A", List.of("B", "C", "D"),
			"B", List.of("E", "F"),
			"C", List.of("G", "H"),
			"D", List.of("I", "J"));

	String current; // 現在の位置

	MazeWorld(String current) {
		this.current = current;
	}

	public boolean isValid() {
		return true;
	}

	public boolean isGoal() {
		return "E".equals(this.current);
	}

	public List<Action> actions() {
		return this.map.getOrDefault(current, Collections.emptyList()).stream()
				.map(p -> (Action) new MazeAction(p))
				.toList();
	}

	public World successor(Action action) {
		var a = (MazeAction) action;
		return new MazeWorld(a.next);
	}

	public String toString() {
		return this.current;
	}
}


\end{screen}

\subsection{実行結果} 

\begin{figure}[!hbt]
  \centering
  \includegraphics[bb=0 0 782 352,width=0.6\linewidth]{2-1result.png}
  \caption{実行結果}
  \label{fig:result}
\end{figure}

\paragraph{自己評価/作業時間: } S/6時間

\section{課題2-1}

\begin{screen}
  interface World の各メソッドで実装すべき内容を リスト3から読み取り,レポートにて説明せよ
\end{screen}

\subsection{### **1.** `boolean isValid()`

**役割:** 現在の状態が有効な状態かどうかを判定する

**MazeProblem での実装:**} 
\begin{lstlisting}
		public boolean isValid() {
		return true;
	}
\end{lstlisting}
**説明:**
- 迷路問題では常に `true` を返しており、**すべての状態が有効** であることを示す
- 一般的には、制約条件を満たさない状態を除外する場合に使用

\subsection{### **2. `boolean isGoal()`**

**役割:** 現在の状態がゴール状態であるかを判定する

**MazeProblem での実装:**} 
\begin{lstlisting}
	public boolean isGoal() {
		return "E".equals(this.current);
	}
\end{lstlisting}

**説明:**

- 現在地 (current) が `"E"` と一致するかをチェック
- ゴール到達時に探索を終了
- 迷路問題では、スタート地点 `"A"` からゴール地点 `"E"` への経路を探索

\subsection{### **3. List<Action> actions()**

**役割:** 現在の状態から実行可能なすべてのアクション(移動先)を返す

**MazeProblem での実装:**}
\begin{lstlisting}
public List<Action> actions() {
    return this.map.getOrDefault(current, Collections.emptyList()).stream()
            .map(p -> (Action) new MazeAction(p))
            .toList();
}
\end{lstlisting}
**説明:**

- map から現在位置 (current) に対応するノードリストを取得
- 各移動先を MazeAction オブジェクトに変換
- 例:位置 `"A"` の場合、`["B", "C", "D"]` に対応する3つのアクションを返す
- 移動先がない場合は空のリストを返す

\subsection{### **4. World successor(Action action)**

**役割:** 指定されたアクションを実行した後の新しい状態(後続状態)を返す

**MazeProblem での実装:**}
\begin{lstlisting}
public World successor(Action action) {
    var a = (MazeAction) action;
    return new MazeWorld(a.next);
}
\end{lstlisting}
**説明:**

- アクション(移動先)を受け取り、**新しい World オブジェクトを生成**
- 新しいオブジェクトの current を移動先に更新


\subsection{考察・感想}

\paragraph{考察: }
課題2では,探索アルゴリズムをより汎用的で拡張性の高い設計に改良することができた.特に,WorldインターフェースとActionインターフェースの導入により,探索問題の抽象化が実現された.これにより,迷路探索以外の問題(例:パズル,経路最適化など)にも同じSolverクラスを適用可能となり,コードの再利用性が大幅に向上した.

インターフェース設計の観点から分析すると,各メソッドの責任が明確に分離されている.isValid()による制約チェック,isGoal()によるゴール判定,actions()による可能な行動の生成,successor()による状態遷移といった,探索アルゴリズムに必要な基本操作が適切に抽象化されている.この設計により,問題固有の詳細実装とアルゴリズムのロジックが分離され,保守性と理解しやすさが向上した.

また,課題1-1aの単純な実装と比較すると,Stateクラスの導入により探索経路の追跡が可能になった点も重要である.parent参照を持つことで,ゴール到達時に完全な解の経路を再構築できるようになり,より実用的なソリューションとなった.

\paragraph{感想: }
課題2の実装を通じて,オブジェクト指向設計の威力を実感することができた.最初は課題1-1aの単純な実装で十分だと思っていたが,インターフェースを導入することで,同じ探索エンジンで全く異なる問題を解けるようになることに感動した.特に,MazeActionやMazeWorldクラスで問題固有の詳細を隠蔽し,Solverクラスで汎用的な探索ロジックを提供するという設計パターンは,今後の開発で活用したい重要な学びとなった.

プログラムを実際に動かしてみると,解の経路が「A → B (move to B) → E (move to E)」のように表示され,どのような手順でゴールに到達したかが明確に分かる点も印象的だった.これは実用的なアプリケーションでは必須の機能であり,理論だけでなく実装面での完成度の重要性を学んだ.今後はさらに複雑な探索問題や,A*アルゴリズムなどの高度な手法にも挑戦してみたい.

%%%%%%%%%%%%%%%%%%%%%%%%%%%%%%%%%%%%%%%%
\section{課題1-1c}

\paragraph{自己評価/作業時間: } S/7時間

\section{課題3-1}

\begin{screen}
 プログラムを実行し,手数(船の移動回数)を確認し,その手数が最小であるかを考察すること.
\end{screen}

\begin{figure}[!hbt]
  \centering
  \includegraphics[bb=0 0 782 352,width=0.6\linewidth]{3-1result.png}
  \caption{実行結果}
  \label{fig:result}
\end{figure}

\section{課題3-2}

\begin{screen}
人間にとって視認しやすい実行例となるよう,プログラムを改良せよ.改良された実行結果をレポートに示せ
\end{screen}

\begin{figure}[!hbt]
  \centering
  \includegraphics[bb=0 0 782 352,width=0.6\linewidth]{3-2result.png}
  \caption{実行結果}
  \label{fig:result}
\end{figure}

\section{課題3-3}

\begin{screen}
探索完了時の訪問ノード数,オープンリストの最大長,実行時間(ミリ秒)を表示する機能を追加せよ.これらの機能を用いて,横型探索および縦型探索の性能を計測し,その結果をレポートにて報告すること
\end{screen}

Search Type           : 縦型探索
Visited Nodes         : 22
Max Open List Size    : 9
Execution Time        : 5 ms

Search Type           : 横型探索
Visited Nodes         : 28
Max Open List Size    : 6
Execution Time        : 59 ms

\subsection{考察・感想}

\paragraph{考察: }
課題1-1cでは,宣教師と人食い人種の問題を実装し,幅優先探索(横型探索)と深さ優先探索(縦型探索)の性能を比較した.実験結果から,縦型探索の訪問ノード数は22,横型探索は28となり,縦型探索の方が効率的に見える.しかし,横型探索が見つけた解は最短手数(11手)であることが保証されている点が重要である.

探索アルゴリズムの性能を分析すると,訪問ノード数だけでなく,オープンリストの最大長や実行時間も考慮する必要がある.横型探索のオープンリスト最大長は6,縦型探索は9となっており,横型探索の方がメモリ効率が良い.実行時間については,縦型探索が5ms,横型探索が59msと大きな差があるが,これは問題規模が小さいため,実際の差というより計測誤差の可能性もある.

また,問題の特性として制約条件(どちらの岸でも人食い人種の数が宣教師の数を上回ってはならない)が厳しく,多くの状態が無効となる.isValid()メソッドでこれらの無効状態を適切にフィルタリングすることで,探索空間を削減できている点も重要である.


\paragraph{感想: }
宣教師と人食い人種の問題は,迷路探索よりも複雑な制約条件を持つため,実装の難易度が高かった.特に,isValid()メソッドで両岸の制約を正しくチェックする部分では,右岸の状態を計算する必要があり(3 - this.missionary,3 - this.cannibal),最初は実装ミスで無限ループに陥ることがあった.

また,toString()メソッドで視覚的に分かりやすい出力を実装したことで,プログラムの動作確認が容易になった.川の両岸の状態とボートの位置が一目で分かる表示により,解の妥当性を人間が直感的に確認できる点は,実用的なプログラム開発において重要だと感じた.

幅優先探索と深さ優先探索の比較を通じて,アルゴリズムの選択が解の質(最短手数)と計算効率(訪問ノード数,実行時間)のトレードオフになることを実感した.今回の問題では最短解が重要なので幅優先探索が適しているが,問題によっては深さ優先探索の方が有効な場合もあり,問題の特性に応じた適切なアルゴリズム選択の重要性を学んだ.

\section{課題1-1d}

\paragraph{自己評価/作業時間: } S/5時間

\section{課題4-1}

\begin{screen}
 まずは,プログラムを使わずに k と処理時間の関係を予想し,そのグラフの概形を示しなさい.
\end{screen}

\begin{figure}[!hbt]
  \centering
  \includegraphics[bb=0 0 782 352,width=0.6\linewidth]{4-1result.png}
  \caption{実行結果}
  \label{fig:result}
\end{figure}

\section{課題4-2}

\begin{screen}
k の変化に伴う,訪問ノード数,オープンリストの最大長,実行時間(ミリ秒)を調べなさい.
\end{screen}

BFSの場合

k=3

Visited Nodes         : 29
Max Open List Size    : 6
Execution Time        : 30 ms

k=4

Visited Nodes         : 52
Max Open List Size    : 12
Execution Time        : 29 ms

k=5

Visited Nodes         : 62
Max Open List Size    : 13
Execution Time        : 37 ms

k=6

Visited Nodes         : 101
Max Open List Size    : 25
Execution Time        : 35 ms

k=7

Visited Nodes         : 121
Max Open List Size    : 28
Execution Time        : 35 ms

k=8

Visited Nodes         : 162
Max Open List Size    : 40
Execution Time        : 60 ms

k=9

Visited Nodes         : 186
Max Open List Size    : 47
Execution Time        : 56 ms

k=10

Visited Nodes         : 247
Max Open List Size    : 66
Execution Time        : 68 ms

DFSの場合

k=3

Visited Nodes         : 17
Max Open List Size    : 10
Execution Time        : 0 ms

k=4

Visited Nodes         : 19
Max Open List Size    : 17
Execution Time        : 7 ms

k=5

Visited Nodes         : 23
Max Open List Size    : 27
Execution Time        : 8 ms

k=6

Visited Nodes         : 35
Max Open List Size    : 51
Execution Time        : 9 ms

k=7

Visited Nodes         : 33
Max Open List Size    : 49
Execution Time        : 7 ms

k=8

Visited Nodes         : 39
Max Open List Size    : 84
Execution Time        : 8 ms

k=9

Visited Nodes         : 37
Max Open List Size    : 80
Execution Time        : 0 ms

k=10

Visited Nodes         : 48
Max Open List Size    : 117
Execution Time        : 15 ms

\section{課題4-3}

\begin{screen}
最後に,k の変化に伴う実行時間等の変化について予想を踏まえて考察しなさい.
\end{screen}

BFSもDFSもkの変化によって実行時間は比例して変化すると思う

\section{課題4-4}

\begin{screen}
クローズドリストを導入し,その効果を検証したうえで,その効果について考察すること.
\end{screen}

すでにクローズドテストを導入していたためベンチマークは上記と変わらない。
クローズドテストの効果比較は
| k |  With Closed List   | Without Closed List | Visited Nodes Ratio | Time Ratio |
|---|---------------------|---------------------|---------------------|------------|
|   | Visited | Time(ms)  | Visited | Time(ms)  | (Without/With)      | (W/o / W)  |
| 3 |      29 |        0  |  10,964 |    1,104  |      378.07         |  1,104.00  |
| 4 |      52 |        0  |  16,806 |    1,864  |      323.19         |  1,864.00

\subsection{考察・感想}

\paragraph{考察: }
課題1-1dでは,k人の宣教師とk人の人食い人種という一般化された問題を実装し,kの変化に伴う性能の変化を測定した.また,クローズドリストの効果を検証することで,探索アルゴリズムの最適化手法について深く理解することができた.

まず,kの増加に伴う性能の変化について分析する.BFSの結果から,k=3では訪問ノード数29,k=10では247と,kが増加するにつれて訪問ノード数が増加している.これは状態空間がkの増加とともに拡大するためである.しかし,増加率は指数関数的ではなく,比較的緩やかな増加に留まっている.これは,制約条件(M >= C)により多くの無効な状態が排除されるためと考えられる.

DFSの場合,訪問ノード数はBFSよりも少ない傾向にあるが(k=3で17,k=10で48),これは運良く早期に解を発見できた場合の結果である.ただし,DFSは最短解を保証しないため,実用的にはBFSの方が信頼性が高い.

最も重要な発見は,クローズドリストの効果である.実験結果から,k=3でクローズドリストなしの場合,訪問ノード数が10,964(約378倍),実行時間が1,104ms(約1,100倍)となった.k=4では訪問ノード数が16,806(約323倍)と,クローズドリストの有無で劇的な差が生じた.これは,同じ状態に到達する経路が複数存在するこの問題の性質により,クローズドリストなしでは同じ状態を何度も訪問してしまうためである.

また,ボート容量を k/2+1 とすることで,kが大きい場合でも解が存在することを確認できた.この設定により,効率的な移動が可能となり,問題の拡張性が向上している.


\paragraph{感想: }
課題1-1dの実装を通じて,アルゴリズムの最適化がいかに重要かを実感した.クローズドリストという単純な改良により,訪問ノード数が約300倍以上削減されるという結果は驚きであった.特に,k=5以上ではクローズドリストなしでは現実的な時間で解を得ることが困難になる点が印象的だった.

また,k=3からk=10まで段階的に性能を測定することで,問題規模の増加に伴うアルゴリズムの振る舞いを可視化できた点も有益だった.グラフや表形式で結果を整理することで,訪問ノード数やオープンリストの最大長がkに対してほぼ線形に増加することが明確になった.

BFSとDFSの比較からは,最短解を求める必要がある場合にはBFSが適していることを再確認できた.DFSは訪問ノード数が少なく見えるが,これは偶然早期に解を発見できた場合であり,常に効率的とは限らない.問題の特性に応じた適切なアルゴリズム選択の重要性を,実データをもとに学ぶことができた.

\section{課題1-2a}

\begin{screen}
演習 1-1 のプログラムにコストを導入するための拡張点について説明せよ,特に,経路コスト g やヒューリスティック関数 h′ の実現方法を答えよ
\end{screen}

船の左右の移動で消費コストが変わったり、船で運ぶ人数でコストが変わったりしてもコストを導入することで対応可能である。この点で拡張性がある。
経路コストgはアクションのコストと状態の累計コストで構成されている。

ヒューリスティック関数h’は
\begin{screen}
class MisCanHeuristic implements Heuristic {
    public float eval(State s) {
        var w = (MisCanWorld) s.world();
        return w.missionary;
    }
}
\end{screen}
実際に左岸にいる宣教師の数をそのまま返しています。

\begin{screen}
どのように探索プログラムに評価関数を与えているのかを説明せよ
\end{screen}
\begin{screen}
public interface Evaluator {
    public static Evaluator minCost() { return new MinCostEvaluator(); }
    public static Evaluator bestFirst(Heuristic h) { return new BestFirstEvaluator(h); }
    public static Evaluator aStar(Heuristic h) { return new AStarEvaluator(h); }

    float f(State s);

    default float g(State s) {
        return s.cost;
    }
}
\end{screen}
この関数によりかかったコストを評価として扱っています。

\section{課題1-2b}

\begin{screen}
最小コスト優先探索
\end{screen}

visited: 4674, max length: 2677
Execution time: 234 ms

\begin{screen}
最良優先探索
\end{screen}

visited: 404, max length: 252
Execution time: 37 ms

\begin{screen}
A*アルゴリズム
\end{screen}

visited: 236, max length: 160
Execution time: 15 ms

\section{課題1-2c}
\begin{screen}
繰り返し回避の導入に伴う訪問ノード数の削減率を報告せよ
\end{screen}

\section{課題1-2d}
\begin{screen}
異なる 3 種類のヒューリスティック関数 h′1, h′2, h′3 を比較し,ヒューリスティック関数の違いが性能に与える影響を調べよ
\end{screen}



\end{document}
